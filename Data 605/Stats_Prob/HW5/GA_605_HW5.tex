% Options for packages loaded elsewhere
\PassOptionsToPackage{unicode}{hyperref}
\PassOptionsToPackage{hyphens}{url}
%
\documentclass[
]{article}
\usepackage{lmodern}
\usepackage{amssymb,amsmath}
\usepackage{ifxetex,ifluatex}
\ifnum 0\ifxetex 1\fi\ifluatex 1\fi=0 % if pdftex
  \usepackage[T1]{fontenc}
  \usepackage[utf8]{inputenc}
  \usepackage{textcomp} % provide euro and other symbols
\else % if luatex or xetex
  \usepackage{unicode-math}
  \defaultfontfeatures{Scale=MatchLowercase}
  \defaultfontfeatures[\rmfamily]{Ligatures=TeX,Scale=1}
\fi
% Use upquote if available, for straight quotes in verbatim environments
\IfFileExists{upquote.sty}{\usepackage{upquote}}{}
\IfFileExists{microtype.sty}{% use microtype if available
  \usepackage[]{microtype}
  \UseMicrotypeSet[protrusion]{basicmath} % disable protrusion for tt fonts
}{}
\makeatletter
\@ifundefined{KOMAClassName}{% if non-KOMA class
  \IfFileExists{parskip.sty}{%
    \usepackage{parskip}
  }{% else
    \setlength{\parindent}{0pt}
    \setlength{\parskip}{6pt plus 2pt minus 1pt}}
}{% if KOMA class
  \KOMAoptions{parskip=half}}
\makeatother
\usepackage{xcolor}
\IfFileExists{xurl.sty}{\usepackage{xurl}}{} % add URL line breaks if available
\IfFileExists{bookmark.sty}{\usepackage{bookmark}}{\usepackage{hyperref}}
\hypersetup{
  pdftitle={GA\_605\_HW5},
  pdfauthor={Gabe Abreu},
  hidelinks,
  pdfcreator={LaTeX via pandoc}}
\urlstyle{same} % disable monospaced font for URLs
\usepackage[margin=1in]{geometry}
\usepackage{color}
\usepackage{fancyvrb}
\newcommand{\VerbBar}{|}
\newcommand{\VERB}{\Verb[commandchars=\\\{\}]}
\DefineVerbatimEnvironment{Highlighting}{Verbatim}{commandchars=\\\{\}}
% Add ',fontsize=\small' for more characters per line
\usepackage{framed}
\definecolor{shadecolor}{RGB}{248,248,248}
\newenvironment{Shaded}{\begin{snugshade}}{\end{snugshade}}
\newcommand{\AlertTok}[1]{\textcolor[rgb]{0.94,0.16,0.16}{#1}}
\newcommand{\AnnotationTok}[1]{\textcolor[rgb]{0.56,0.35,0.01}{\textbf{\textit{#1}}}}
\newcommand{\AttributeTok}[1]{\textcolor[rgb]{0.77,0.63,0.00}{#1}}
\newcommand{\BaseNTok}[1]{\textcolor[rgb]{0.00,0.00,0.81}{#1}}
\newcommand{\BuiltInTok}[1]{#1}
\newcommand{\CharTok}[1]{\textcolor[rgb]{0.31,0.60,0.02}{#1}}
\newcommand{\CommentTok}[1]{\textcolor[rgb]{0.56,0.35,0.01}{\textit{#1}}}
\newcommand{\CommentVarTok}[1]{\textcolor[rgb]{0.56,0.35,0.01}{\textbf{\textit{#1}}}}
\newcommand{\ConstantTok}[1]{\textcolor[rgb]{0.00,0.00,0.00}{#1}}
\newcommand{\ControlFlowTok}[1]{\textcolor[rgb]{0.13,0.29,0.53}{\textbf{#1}}}
\newcommand{\DataTypeTok}[1]{\textcolor[rgb]{0.13,0.29,0.53}{#1}}
\newcommand{\DecValTok}[1]{\textcolor[rgb]{0.00,0.00,0.81}{#1}}
\newcommand{\DocumentationTok}[1]{\textcolor[rgb]{0.56,0.35,0.01}{\textbf{\textit{#1}}}}
\newcommand{\ErrorTok}[1]{\textcolor[rgb]{0.64,0.00,0.00}{\textbf{#1}}}
\newcommand{\ExtensionTok}[1]{#1}
\newcommand{\FloatTok}[1]{\textcolor[rgb]{0.00,0.00,0.81}{#1}}
\newcommand{\FunctionTok}[1]{\textcolor[rgb]{0.00,0.00,0.00}{#1}}
\newcommand{\ImportTok}[1]{#1}
\newcommand{\InformationTok}[1]{\textcolor[rgb]{0.56,0.35,0.01}{\textbf{\textit{#1}}}}
\newcommand{\KeywordTok}[1]{\textcolor[rgb]{0.13,0.29,0.53}{\textbf{#1}}}
\newcommand{\NormalTok}[1]{#1}
\newcommand{\OperatorTok}[1]{\textcolor[rgb]{0.81,0.36,0.00}{\textbf{#1}}}
\newcommand{\OtherTok}[1]{\textcolor[rgb]{0.56,0.35,0.01}{#1}}
\newcommand{\PreprocessorTok}[1]{\textcolor[rgb]{0.56,0.35,0.01}{\textit{#1}}}
\newcommand{\RegionMarkerTok}[1]{#1}
\newcommand{\SpecialCharTok}[1]{\textcolor[rgb]{0.00,0.00,0.00}{#1}}
\newcommand{\SpecialStringTok}[1]{\textcolor[rgb]{0.31,0.60,0.02}{#1}}
\newcommand{\StringTok}[1]{\textcolor[rgb]{0.31,0.60,0.02}{#1}}
\newcommand{\VariableTok}[1]{\textcolor[rgb]{0.00,0.00,0.00}{#1}}
\newcommand{\VerbatimStringTok}[1]{\textcolor[rgb]{0.31,0.60,0.02}{#1}}
\newcommand{\WarningTok}[1]{\textcolor[rgb]{0.56,0.35,0.01}{\textbf{\textit{#1}}}}
\usepackage{graphicx,grffile}
\makeatletter
\def\maxwidth{\ifdim\Gin@nat@width>\linewidth\linewidth\else\Gin@nat@width\fi}
\def\maxheight{\ifdim\Gin@nat@height>\textheight\textheight\else\Gin@nat@height\fi}
\makeatother
% Scale images if necessary, so that they will not overflow the page
% margins by default, and it is still possible to overwrite the defaults
% using explicit options in \includegraphics[width, height, ...]{}
\setkeys{Gin}{width=\maxwidth,height=\maxheight,keepaspectratio}
% Set default figure placement to htbp
\makeatletter
\def\fps@figure{htbp}
\makeatother
\setlength{\emergencystretch}{3em} % prevent overfull lines
\providecommand{\tightlist}{%
  \setlength{\itemsep}{0pt}\setlength{\parskip}{0pt}}
\setcounter{secnumdepth}{-\maxdimen} % remove section numbering

\title{GA\_605\_HW5}
\author{Gabe Abreu}
\date{9/27/2020}

\begin{document}
\maketitle

\hypertarget{assignment-week-5}{%
\subsection{Assignment Week 5}\label{assignment-week-5}}

Choose independently two numbers B and C at random from the interval
{[}0, 1{]} with uniform density. Prove that B and C are proper
probability distributions. Note that the point (B,C) is then chosen at
random in the unit square. Find the probability that

\begin{Shaded}
\begin{Highlighting}[]
\CommentTok{#Create the variables b and c, using the function runif}
\CommentTok{#runif will run 1000, and not reach the min or max }
\NormalTok{b <-}\StringTok{ }\KeywordTok{runif}\NormalTok{(}\DecValTok{1000}\NormalTok{, }\DataTypeTok{min =} \DecValTok{0}\NormalTok{, }\DataTypeTok{max =} \DecValTok{1}\NormalTok{)}
\NormalTok{c <-}\StringTok{ }\KeywordTok{runif}\NormalTok{(}\DecValTok{1000}\NormalTok{, }\DataTypeTok{min =} \DecValTok{0}\NormalTok{, }\DataTypeTok{max =} \DecValTok{1}\NormalTok{)}

\CommentTok{#Let's use historgrams to check the distributions of the b and c, both are about evenly distributed}
\KeywordTok{hist}\NormalTok{(b)}
\end{Highlighting}
\end{Shaded}

\includegraphics{GA_605_HW5_files/figure-latex/unnamed-chunk-1-1.pdf}

\begin{Shaded}
\begin{Highlighting}[]
\KeywordTok{hist}\NormalTok{(c)}
\end{Highlighting}
\end{Shaded}

\includegraphics{GA_605_HW5_files/figure-latex/unnamed-chunk-1-2.pdf}

B + C \textless{} 1/2

\begin{Shaded}
\begin{Highlighting}[]
\CommentTok{#Sum gives the total number of values that meet the condition, in this case values below 0.5}
\NormalTok{ans_A <-}\StringTok{ }\KeywordTok{sum}\NormalTok{(b }\OperatorTok{+}\StringTok{ }\NormalTok{c }\OperatorTok{<}\StringTok{ }\FloatTok{0.5}\NormalTok{)}

\CommentTok{#Divide by 1000 to calculate the probability}
\NormalTok{ans_A <-}\StringTok{ }\NormalTok{ans_A}\OperatorTok{/}\DecValTok{1000}

\KeywordTok{paste}\NormalTok{(}\StringTok{"The probability is"}\NormalTok{, ans_A)}
\end{Highlighting}
\end{Shaded}

\begin{verbatim}
## [1] "The probability is 0.141"
\end{verbatim}

BC \textless{} 1/2

\begin{Shaded}
\begin{Highlighting}[]
\NormalTok{ans_B <-}\StringTok{ }\KeywordTok{sum}\NormalTok{(b}\OperatorTok{*}\NormalTok{c }\OperatorTok{<}\StringTok{ }\FloatTok{0.5}\NormalTok{)}\OperatorTok{/}\DecValTok{1000}
\KeywordTok{paste}\NormalTok{(}\StringTok{"The probability is"}\NormalTok{, ans_B)}
\end{Highlighting}
\end{Shaded}

\begin{verbatim}
## [1] "The probability is 0.823"
\end{verbatim}

\textbar B − C\textbar{} \textless{} 1/2

\begin{Shaded}
\begin{Highlighting}[]
\NormalTok{ans_C <-}\StringTok{ }\KeywordTok{sum}\NormalTok{(}\KeywordTok{abs}\NormalTok{(b}\OperatorTok{-}\NormalTok{c) }\OperatorTok{<}\StringTok{ }\FloatTok{0.5}\NormalTok{)}\OperatorTok{/}\DecValTok{1000}
\KeywordTok{paste}\NormalTok{(}\StringTok{"The probability is"}\NormalTok{, ans_C)}
\end{Highlighting}
\end{Shaded}

\begin{verbatim}
## [1] "The probability is 0.752"
\end{verbatim}

max\{B,C\} \textless{} 1/2

\begin{Shaded}
\begin{Highlighting}[]
\CommentTok{#Used Max function and the answer was .001, so used the pmax and pmin functions}
\NormalTok{ans_D <-}\StringTok{ }\KeywordTok{sum}\NormalTok{(}\KeywordTok{pmax}\NormalTok{(b,c) }\OperatorTok{<}\StringTok{ }\FloatTok{0.5}\NormalTok{)}\OperatorTok{/}\DecValTok{1000}
\KeywordTok{paste}\NormalTok{(}\StringTok{"The probability is"}\NormalTok{, ans_D)}
\end{Highlighting}
\end{Shaded}

\begin{verbatim}
## [1] "The probability is 0.25"
\end{verbatim}

min\{B,C\} \textless{} 1/2

\begin{Shaded}
\begin{Highlighting}[]
\NormalTok{ans_E <-}\StringTok{ }\KeywordTok{sum}\NormalTok{(}\KeywordTok{pmin}\NormalTok{(b,c) }\OperatorTok{<}\StringTok{ }\FloatTok{0.5}\NormalTok{)}\OperatorTok{/}\DecValTok{1000}
\KeywordTok{paste}\NormalTok{(}\StringTok{"The probability is"}\NormalTok{, ans_E)}
\end{Highlighting}
\end{Shaded}

\begin{verbatim}
## [1] "The probability is 0.754"
\end{verbatim}

\end{document}
