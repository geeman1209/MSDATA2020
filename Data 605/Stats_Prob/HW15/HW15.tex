% Options for packages loaded elsewhere
\PassOptionsToPackage{unicode}{hyperref}
\PassOptionsToPackage{hyphens}{url}
%
\documentclass[
]{article}
\usepackage{lmodern}
\usepackage{amssymb,amsmath}
\usepackage{ifxetex,ifluatex}
\ifnum 0\ifxetex 1\fi\ifluatex 1\fi=0 % if pdftex
  \usepackage[T1]{fontenc}
  \usepackage[utf8]{inputenc}
  \usepackage{textcomp} % provide euro and other symbols
\else % if luatex or xetex
  \usepackage{unicode-math}
  \defaultfontfeatures{Scale=MatchLowercase}
  \defaultfontfeatures[\rmfamily]{Ligatures=TeX,Scale=1}
\fi
% Use upquote if available, for straight quotes in verbatim environments
\IfFileExists{upquote.sty}{\usepackage{upquote}}{}
\IfFileExists{microtype.sty}{% use microtype if available
  \usepackage[]{microtype}
  \UseMicrotypeSet[protrusion]{basicmath} % disable protrusion for tt fonts
}{}
\makeatletter
\@ifundefined{KOMAClassName}{% if non-KOMA class
  \IfFileExists{parskip.sty}{%
    \usepackage{parskip}
  }{% else
    \setlength{\parindent}{0pt}
    \setlength{\parskip}{6pt plus 2pt minus 1pt}}
}{% if KOMA class
  \KOMAoptions{parskip=half}}
\makeatother
\usepackage{xcolor}
\IfFileExists{xurl.sty}{\usepackage{xurl}}{} % add URL line breaks if available
\IfFileExists{bookmark.sty}{\usepackage{bookmark}}{\usepackage{hyperref}}
\hypersetup{
  pdftitle={Data605\_HW15},
  pdfauthor={Gabe Abreu},
  hidelinks,
  pdfcreator={LaTeX via pandoc}}
\urlstyle{same} % disable monospaced font for URLs
\usepackage[margin=1in]{geometry}
\usepackage{color}
\usepackage{fancyvrb}
\newcommand{\VerbBar}{|}
\newcommand{\VERB}{\Verb[commandchars=\\\{\}]}
\DefineVerbatimEnvironment{Highlighting}{Verbatim}{commandchars=\\\{\}}
% Add ',fontsize=\small' for more characters per line
\usepackage{framed}
\definecolor{shadecolor}{RGB}{248,248,248}
\newenvironment{Shaded}{\begin{snugshade}}{\end{snugshade}}
\newcommand{\AlertTok}[1]{\textcolor[rgb]{0.94,0.16,0.16}{#1}}
\newcommand{\AnnotationTok}[1]{\textcolor[rgb]{0.56,0.35,0.01}{\textbf{\textit{#1}}}}
\newcommand{\AttributeTok}[1]{\textcolor[rgb]{0.77,0.63,0.00}{#1}}
\newcommand{\BaseNTok}[1]{\textcolor[rgb]{0.00,0.00,0.81}{#1}}
\newcommand{\BuiltInTok}[1]{#1}
\newcommand{\CharTok}[1]{\textcolor[rgb]{0.31,0.60,0.02}{#1}}
\newcommand{\CommentTok}[1]{\textcolor[rgb]{0.56,0.35,0.01}{\textit{#1}}}
\newcommand{\CommentVarTok}[1]{\textcolor[rgb]{0.56,0.35,0.01}{\textbf{\textit{#1}}}}
\newcommand{\ConstantTok}[1]{\textcolor[rgb]{0.00,0.00,0.00}{#1}}
\newcommand{\ControlFlowTok}[1]{\textcolor[rgb]{0.13,0.29,0.53}{\textbf{#1}}}
\newcommand{\DataTypeTok}[1]{\textcolor[rgb]{0.13,0.29,0.53}{#1}}
\newcommand{\DecValTok}[1]{\textcolor[rgb]{0.00,0.00,0.81}{#1}}
\newcommand{\DocumentationTok}[1]{\textcolor[rgb]{0.56,0.35,0.01}{\textbf{\textit{#1}}}}
\newcommand{\ErrorTok}[1]{\textcolor[rgb]{0.64,0.00,0.00}{\textbf{#1}}}
\newcommand{\ExtensionTok}[1]{#1}
\newcommand{\FloatTok}[1]{\textcolor[rgb]{0.00,0.00,0.81}{#1}}
\newcommand{\FunctionTok}[1]{\textcolor[rgb]{0.00,0.00,0.00}{#1}}
\newcommand{\ImportTok}[1]{#1}
\newcommand{\InformationTok}[1]{\textcolor[rgb]{0.56,0.35,0.01}{\textbf{\textit{#1}}}}
\newcommand{\KeywordTok}[1]{\textcolor[rgb]{0.13,0.29,0.53}{\textbf{#1}}}
\newcommand{\NormalTok}[1]{#1}
\newcommand{\OperatorTok}[1]{\textcolor[rgb]{0.81,0.36,0.00}{\textbf{#1}}}
\newcommand{\OtherTok}[1]{\textcolor[rgb]{0.56,0.35,0.01}{#1}}
\newcommand{\PreprocessorTok}[1]{\textcolor[rgb]{0.56,0.35,0.01}{\textit{#1}}}
\newcommand{\RegionMarkerTok}[1]{#1}
\newcommand{\SpecialCharTok}[1]{\textcolor[rgb]{0.00,0.00,0.00}{#1}}
\newcommand{\SpecialStringTok}[1]{\textcolor[rgb]{0.31,0.60,0.02}{#1}}
\newcommand{\StringTok}[1]{\textcolor[rgb]{0.31,0.60,0.02}{#1}}
\newcommand{\VariableTok}[1]{\textcolor[rgb]{0.00,0.00,0.00}{#1}}
\newcommand{\VerbatimStringTok}[1]{\textcolor[rgb]{0.31,0.60,0.02}{#1}}
\newcommand{\WarningTok}[1]{\textcolor[rgb]{0.56,0.35,0.01}{\textbf{\textit{#1}}}}
\usepackage{graphicx,grffile}
\makeatletter
\def\maxwidth{\ifdim\Gin@nat@width>\linewidth\linewidth\else\Gin@nat@width\fi}
\def\maxheight{\ifdim\Gin@nat@height>\textheight\textheight\else\Gin@nat@height\fi}
\makeatother
% Scale images if necessary, so that they will not overflow the page
% margins by default, and it is still possible to overwrite the defaults
% using explicit options in \includegraphics[width, height, ...]{}
\setkeys{Gin}{width=\maxwidth,height=\maxheight,keepaspectratio}
% Set default figure placement to htbp
\makeatletter
\def\fps@figure{htbp}
\makeatother
\setlength{\emergencystretch}{3em} % prevent overfull lines
\providecommand{\tightlist}{%
  \setlength{\itemsep}{0pt}\setlength{\parskip}{0pt}}
\setcounter{secnumdepth}{-\maxdimen} % remove section numbering
\usepackage{bbm}

\title{Data605\_HW15}
\author{Gabe Abreu}
\date{12/9/2020}

\begin{document}
\maketitle

\hypertarget{data-605-hw-15}{%
\subsection{Data 605 HW \#15}\label{data-605-hw-15}}

\begin{enumerate}
\def\labelenumi{\arabic{enumi}.}
\tightlist
\item
  Find the equation of the regression line for the given points. Round
  any final values to the nearest hundredth, if necessary. ( 5.6, 8.8 ),
  ( 6.3, 12.4 ), ( 7, 14.8 ), ( 7.7, 18.2 ), ( 8.4, 20.8 )
\end{enumerate}

\begin{Shaded}
\begin{Highlighting}[]
\NormalTok{pt1 <-}\StringTok{ }\KeywordTok{c}\NormalTok{(}\FloatTok{5.6}\NormalTok{, }\FloatTok{6.3}\NormalTok{, }\DecValTok{7}\NormalTok{, }\FloatTok{7.7}\NormalTok{, }\FloatTok{8.4}\NormalTok{)}
\NormalTok{pt2 <-}\StringTok{ }\KeywordTok{c}\NormalTok{(}\FloatTok{8.8}\NormalTok{, }\FloatTok{12.4}\NormalTok{, }\FloatTok{14.8}\NormalTok{, }\FloatTok{18.2}\NormalTok{, }\FloatTok{20.8}\NormalTok{)}

\NormalTok{reg <-}\StringTok{ }\KeywordTok{lm}\NormalTok{(pt2 }\OperatorTok{~}\StringTok{ }\NormalTok{pt1)}
\KeywordTok{summary}\NormalTok{(reg)}
\end{Highlighting}
\end{Shaded}

\begin{verbatim}
## 
## Call:
## lm(formula = pt2 ~ pt1)
## 
## Residuals:
##     1     2     3     4     5 
## -0.24  0.38 -0.20  0.22 -0.16 
## 
## Coefficients:
##             Estimate Std. Error t value Pr(>|t|)    
## (Intercept) -14.8000     1.0365  -14.28 0.000744 ***
## pt1           4.2571     0.1466   29.04 8.97e-05 ***
## ---
## Signif. codes:  0 '***' 0.001 '**' 0.01 '*' 0.05 '.' 0.1 ' ' 1
## 
## Residual standard error: 0.3246 on 3 degrees of freedom
## Multiple R-squared:  0.9965, Adjusted R-squared:  0.9953 
## F-statistic: 843.1 on 1 and 3 DF,  p-value: 8.971e-05
\end{verbatim}

y = -14.8 + 4.2571x

\begin{enumerate}
\def\labelenumi{\arabic{enumi}.}
\setcounter{enumi}{1}
\tightlist
\item
  Find all local maxima, local minima, and saddle points for the
  function given below. Write your answer(s) in the form ( x, y, z ).
  Separate multiple points with a comma
\end{enumerate}

f(x,y) = 24x - 6xy\^{}\{2\} - 8y\^{}\{3\} \textbackslash\textbackslash{}
Partial Derivatives \textbackslash{} f\_\{x\}(x,y) = 24 - 6y\^{}\{2\}
\textbackslash{}\\
f\_\{y\}(x,y) = -12xy - 24y\^{}\{2\} \textbackslash\textbackslash{} X =
0 \textbackslash{} 24 - 6y\^{}\{2\} = 0 \textbackslash{} y = +-2
\textbackslash\textbackslash{} y=2 \textbackslash{} -12x\emph{2 - 24 } 4
\textbackslash{} x = -4 \textbackslash{} y = -2 \textbackslash{} -12x *
-2 - 24 * 4 \textbackslash{} x = 2 \textbackslash\textbackslash{}
Critical Points \textbackslash{} (-4, 2)(4, -2)

\begin{enumerate}
\def\labelenumi{\arabic{enumi}.}
\setcounter{enumi}{2}
\tightlist
\item
  A grocery store sells two brands of a product, the ``house'' brand and
  a ``name'' brand. The manager estimates that if she sells the
  ``house'' brand for x dollars and the ``name'' brand for y dollars,
  she will be able to sell 81  21x + 17y units of the ``house'' brand
  and 40 + 11x  23y units of the ``name'' brand.
\end{enumerate}

Step 1. Find the revenue function R ( x, y ).

\begin{Shaded}
\begin{Highlighting}[]
\NormalTok{revenue_func <-}\StringTok{ }\ControlFlowTok{function}\NormalTok{(x, y)\{}
\NormalTok{  x}\OperatorTok{*}\NormalTok{(}\DecValTok{81} \OperatorTok{-}\StringTok{ }\NormalTok{(}\DecValTok{21}\OperatorTok{*}\NormalTok{x) }\OperatorTok{+}\StringTok{ }\NormalTok{(}\DecValTok{17}\OperatorTok{*}\NormalTok{y)) }\OperatorTok{+}\StringTok{ }\NormalTok{y }\OperatorTok{*}\StringTok{ }\NormalTok{(}\DecValTok{40} \OperatorTok{+}\StringTok{ }\NormalTok{(}\DecValTok{11} \OperatorTok{*}\StringTok{ }\NormalTok{x) }\OperatorTok{-}\StringTok{ }\DecValTok{23}\OperatorTok{*}\NormalTok{y)}
\NormalTok{\}}
\end{Highlighting}
\end{Shaded}

Step 2. What is the revenue if she sells the ``house'' brand for \$2.30
and the ``name'' brand for \$4.10?

\begin{Shaded}
\begin{Highlighting}[]
\KeywordTok{revenue_func}\NormalTok{(}\FloatTok{2.3}\NormalTok{, }\FloatTok{4.1}\NormalTok{)}
\end{Highlighting}
\end{Shaded}

\begin{verbatim}
## [1] 116.62
\end{verbatim}

\begin{enumerate}
\def\labelenumi{\arabic{enumi}.}
\setcounter{enumi}{3}
\item
  A company has a plant in Los Angeles and a plant in Denver. The firm
  is committed to produce a total of 96 units of a product each week.
  The total weekly cost is given by C(x, y) = 1/6x\^{}2 + 1/6 y\^{}2 +
  7x + 25y + 700, where x is the number of units produced in Los Angeles
  and y is the number of units produced in Denver. How many units should
  be produced in each plant to minimize the total weekly cost?
\item
  Evaluate the double integral on the given region. Write your answer in
  exact form without decimals.
\end{enumerate}

\begin{Shaded}
\begin{Highlighting}[]
\NormalTok{double_integral <-}\StringTok{ }\ControlFlowTok{function}\NormalTok{(x,y) }\KeywordTok{exp}\NormalTok{(}\DecValTok{8}\OperatorTok{*}\NormalTok{x }\OperatorTok{+}\StringTok{ }\DecValTok{3} \OperatorTok{*}\StringTok{ }\NormalTok{y)}

\KeywordTok{format}\NormalTok{(}\KeywordTok{round}\NormalTok{(}\KeywordTok{quad2d}\NormalTok{(double_integral, }\DecValTok{2}\NormalTok{, }\DecValTok{4}\NormalTok{, }\DecValTok{2}\NormalTok{, }\DecValTok{4}\NormalTok{), }\DecValTok{17}\NormalTok{), }\DataTypeTok{scientific =}  \OtherTok{FALSE}\NormalTok{)}
\end{Highlighting}
\end{Shaded}

\begin{verbatim}
## [1] "534155947497083904"
\end{verbatim}

\end{document}
