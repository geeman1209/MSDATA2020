% Options for packages loaded elsewhere
\PassOptionsToPackage{unicode}{hyperref}
\PassOptionsToPackage{hyphens}{url}
%
\documentclass[
]{article}
\usepackage{lmodern}
\usepackage{amssymb,amsmath}
\usepackage{ifxetex,ifluatex}
\ifnum 0\ifxetex 1\fi\ifluatex 1\fi=0 % if pdftex
  \usepackage[T1]{fontenc}
  \usepackage[utf8]{inputenc}
  \usepackage{textcomp} % provide euro and other symbols
\else % if luatex or xetex
  \usepackage{unicode-math}
  \defaultfontfeatures{Scale=MatchLowercase}
  \defaultfontfeatures[\rmfamily]{Ligatures=TeX,Scale=1}
\fi
% Use upquote if available, for straight quotes in verbatim environments
\IfFileExists{upquote.sty}{\usepackage{upquote}}{}
\IfFileExists{microtype.sty}{% use microtype if available
  \usepackage[]{microtype}
  \UseMicrotypeSet[protrusion]{basicmath} % disable protrusion for tt fonts
}{}
\makeatletter
\@ifundefined{KOMAClassName}{% if non-KOMA class
  \IfFileExists{parskip.sty}{%
    \usepackage{parskip}
  }{% else
    \setlength{\parindent}{0pt}
    \setlength{\parskip}{6pt plus 2pt minus 1pt}}
}{% if KOMA class
  \KOMAoptions{parskip=half}}
\makeatother
\usepackage{xcolor}
\IfFileExists{xurl.sty}{\usepackage{xurl}}{} % add URL line breaks if available
\IfFileExists{bookmark.sty}{\usepackage{bookmark}}{\usepackage{hyperref}}
\hypersetup{
  pdftitle={Discussion 14},
  pdfauthor={Gabe Abreu},
  hidelinks,
  pdfcreator={LaTeX via pandoc}}
\urlstyle{same} % disable monospaced font for URLs
\usepackage[margin=1in]{geometry}
\usepackage{color}
\usepackage{fancyvrb}
\newcommand{\VerbBar}{|}
\newcommand{\VERB}{\Verb[commandchars=\\\{\}]}
\DefineVerbatimEnvironment{Highlighting}{Verbatim}{commandchars=\\\{\}}
% Add ',fontsize=\small' for more characters per line
\usepackage{framed}
\definecolor{shadecolor}{RGB}{248,248,248}
\newenvironment{Shaded}{\begin{snugshade}}{\end{snugshade}}
\newcommand{\AlertTok}[1]{\textcolor[rgb]{0.94,0.16,0.16}{#1}}
\newcommand{\AnnotationTok}[1]{\textcolor[rgb]{0.56,0.35,0.01}{\textbf{\textit{#1}}}}
\newcommand{\AttributeTok}[1]{\textcolor[rgb]{0.77,0.63,0.00}{#1}}
\newcommand{\BaseNTok}[1]{\textcolor[rgb]{0.00,0.00,0.81}{#1}}
\newcommand{\BuiltInTok}[1]{#1}
\newcommand{\CharTok}[1]{\textcolor[rgb]{0.31,0.60,0.02}{#1}}
\newcommand{\CommentTok}[1]{\textcolor[rgb]{0.56,0.35,0.01}{\textit{#1}}}
\newcommand{\CommentVarTok}[1]{\textcolor[rgb]{0.56,0.35,0.01}{\textbf{\textit{#1}}}}
\newcommand{\ConstantTok}[1]{\textcolor[rgb]{0.00,0.00,0.00}{#1}}
\newcommand{\ControlFlowTok}[1]{\textcolor[rgb]{0.13,0.29,0.53}{\textbf{#1}}}
\newcommand{\DataTypeTok}[1]{\textcolor[rgb]{0.13,0.29,0.53}{#1}}
\newcommand{\DecValTok}[1]{\textcolor[rgb]{0.00,0.00,0.81}{#1}}
\newcommand{\DocumentationTok}[1]{\textcolor[rgb]{0.56,0.35,0.01}{\textbf{\textit{#1}}}}
\newcommand{\ErrorTok}[1]{\textcolor[rgb]{0.64,0.00,0.00}{\textbf{#1}}}
\newcommand{\ExtensionTok}[1]{#1}
\newcommand{\FloatTok}[1]{\textcolor[rgb]{0.00,0.00,0.81}{#1}}
\newcommand{\FunctionTok}[1]{\textcolor[rgb]{0.00,0.00,0.00}{#1}}
\newcommand{\ImportTok}[1]{#1}
\newcommand{\InformationTok}[1]{\textcolor[rgb]{0.56,0.35,0.01}{\textbf{\textit{#1}}}}
\newcommand{\KeywordTok}[1]{\textcolor[rgb]{0.13,0.29,0.53}{\textbf{#1}}}
\newcommand{\NormalTok}[1]{#1}
\newcommand{\OperatorTok}[1]{\textcolor[rgb]{0.81,0.36,0.00}{\textbf{#1}}}
\newcommand{\OtherTok}[1]{\textcolor[rgb]{0.56,0.35,0.01}{#1}}
\newcommand{\PreprocessorTok}[1]{\textcolor[rgb]{0.56,0.35,0.01}{\textit{#1}}}
\newcommand{\RegionMarkerTok}[1]{#1}
\newcommand{\SpecialCharTok}[1]{\textcolor[rgb]{0.00,0.00,0.00}{#1}}
\newcommand{\SpecialStringTok}[1]{\textcolor[rgb]{0.31,0.60,0.02}{#1}}
\newcommand{\StringTok}[1]{\textcolor[rgb]{0.31,0.60,0.02}{#1}}
\newcommand{\VariableTok}[1]{\textcolor[rgb]{0.00,0.00,0.00}{#1}}
\newcommand{\VerbatimStringTok}[1]{\textcolor[rgb]{0.31,0.60,0.02}{#1}}
\newcommand{\WarningTok}[1]{\textcolor[rgb]{0.56,0.35,0.01}{\textbf{\textit{#1}}}}
\usepackage{graphicx,grffile}
\makeatletter
\def\maxwidth{\ifdim\Gin@nat@width>\linewidth\linewidth\else\Gin@nat@width\fi}
\def\maxheight{\ifdim\Gin@nat@height>\textheight\textheight\else\Gin@nat@height\fi}
\makeatother
% Scale images if necessary, so that they will not overflow the page
% margins by default, and it is still possible to overwrite the defaults
% using explicit options in \includegraphics[width, height, ...]{}
\setkeys{Gin}{width=\maxwidth,height=\maxheight,keepaspectratio}
% Set default figure placement to htbp
\makeatletter
\def\fps@figure{htbp}
\makeatother
\setlength{\emergencystretch}{3em} % prevent overfull lines
\providecommand{\tightlist}{%
  \setlength{\itemsep}{0pt}\setlength{\parskip}{0pt}}
\setcounter{secnumdepth}{-\maxdimen} % remove section numbering

\title{Discussion 14}
\author{Gabe Abreu}
\date{12/2/2020}

\begin{document}
\maketitle

\hypertarget{chapter-7-16-p.360}{%
\subsection{Chapter 7 \#16 p.360}\label{chapter-7-16-p.360}}

find the total area enclosed by the funcƟons f and g

f(x) = x\^{}3 − 4x\^{}2 + x − 1, g(x) = −x\^{}2 + 2x − 4

\begin{Shaded}
\begin{Highlighting}[]
\CommentTok{#functions f and g }
\NormalTok{f <-}\StringTok{ }\ControlFlowTok{function}\NormalTok{(x) \{x}\OperatorTok{^}\DecValTok{3} \OperatorTok{-}\StringTok{ }\DecValTok{4}\OperatorTok{*}\NormalTok{x}\OperatorTok{^}\DecValTok{2} \OperatorTok{+}\StringTok{ }\NormalTok{x }\OperatorTok{-}\StringTok{ }\DecValTok{1}\NormalTok{\}}
\NormalTok{g <-}\StringTok{ }\ControlFlowTok{function}\NormalTok{(x) \{}\OperatorTok{-}\NormalTok{x}\OperatorTok{^}\DecValTok{2} \OperatorTok{+}\StringTok{ }\DecValTok{2}\OperatorTok{*}\NormalTok{x }\OperatorTok{-}\StringTok{ }\DecValTok{4}\NormalTok{\}}


\CommentTok{#Graphed functions }
\KeywordTok{curve}\NormalTok{(f, }\DecValTok{-2}\NormalTok{, }\DecValTok{4}\NormalTok{)}
\KeywordTok{curve}\NormalTok{(g, }\DecValTok{-2}\NormalTok{, }\DecValTok{4}\NormalTok{, }\DataTypeTok{add=}\NormalTok{T)}
\end{Highlighting}
\end{Shaded}

\includegraphics{Discussion13_Ch7_16_files/figure-latex/unnamed-chunk-1-1.pdf}
Graphically, the roots seem to be -1, 1, and 3.

\begin{Shaded}
\begin{Highlighting}[]
\CommentTok{#Find roots using R}

\NormalTok{root1 <-}\StringTok{ }\KeywordTok{uniroot}\NormalTok{(}\ControlFlowTok{function}\NormalTok{(x) }\KeywordTok{f}\NormalTok{(x) }\OperatorTok{-}\StringTok{ }\KeywordTok{g}\NormalTok{(x), }\DataTypeTok{interval =} \KeywordTok{c}\NormalTok{(}\OperatorTok{-}\DecValTok{2}\NormalTok{, }\DecValTok{0}\NormalTok{))}
\NormalTok{root1}\OperatorTok{$}\NormalTok{root}
\end{Highlighting}
\end{Shaded}

\begin{verbatim}
## [1] -0.9999973
\end{verbatim}

\begin{Shaded}
\begin{Highlighting}[]
\NormalTok{root2 <-}\StringTok{ }\KeywordTok{uniroot}\NormalTok{(}\ControlFlowTok{function}\NormalTok{(x) }\KeywordTok{f}\NormalTok{(x) }\OperatorTok{-}\StringTok{ }\KeywordTok{g}\NormalTok{(x), }\DataTypeTok{interval =} \KeywordTok{c}\NormalTok{(}\DecValTok{0}\NormalTok{, }\DecValTok{2}\NormalTok{))}
\NormalTok{root2}\OperatorTok{$}\NormalTok{root}
\end{Highlighting}
\end{Shaded}

\begin{verbatim}
## [1] 1
\end{verbatim}

\begin{Shaded}
\begin{Highlighting}[]
\NormalTok{root3 <-}\StringTok{ }\KeywordTok{uniroot}\NormalTok{(}\ControlFlowTok{function}\NormalTok{(x) }\KeywordTok{f}\NormalTok{(x) }\OperatorTok{-}\StringTok{ }\KeywordTok{g}\NormalTok{(x), }\DataTypeTok{interval =} \KeywordTok{c}\NormalTok{(}\DecValTok{2}\NormalTok{, }\DecValTok{4}\NormalTok{))}
\NormalTok{root3}\OperatorTok{$}\NormalTok{root}
\end{Highlighting}
\end{Shaded}

\begin{verbatim}
## [1] 2.999997
\end{verbatim}

Finding the area

\begin{Shaded}
\begin{Highlighting}[]
\CommentTok{#first function btwn first interval}
\NormalTok{f1 <-}\StringTok{ }\KeywordTok{integrate}\NormalTok{(f, }\DataTypeTok{lower =} \DecValTok{-1}\NormalTok{, }\DataTypeTok{upper =} \DecValTok{1}\NormalTok{)}

\CommentTok{#second function btwn first interval}
\NormalTok{g1 <-}\StringTok{ }\KeywordTok{integrate}\NormalTok{(g, }\DataTypeTok{lower =} \DecValTok{-1}\NormalTok{, }\DataTypeTok{upper =} \DecValTok{1}\NormalTok{)}

\CommentTok{#############################################}

\NormalTok{f2 <-}\StringTok{ }\KeywordTok{integrate}\NormalTok{(f, }\DataTypeTok{lower =} \DecValTok{1}\NormalTok{, }\DataTypeTok{upper =} \DecValTok{3}\NormalTok{)}

\NormalTok{g2 <-}\StringTok{ }\KeywordTok{integrate}\NormalTok{(g, }\DataTypeTok{lower =} \DecValTok{1}\NormalTok{, }\DataTypeTok{upper =} \DecValTok{3}\NormalTok{)}

\NormalTok{total_area <-}\StringTok{ }\NormalTok{(f1}\OperatorTok{$}\NormalTok{value }\OperatorTok{-}\StringTok{ }\NormalTok{g1}\OperatorTok{$}\NormalTok{value) }\OperatorTok{+}\StringTok{ }\NormalTok{(g2}\OperatorTok{$}\NormalTok{value }\OperatorTok{-}\StringTok{ }\NormalTok{f2}\OperatorTok{$}\NormalTok{value)}
\NormalTok{total_area}
\end{Highlighting}
\end{Shaded}

\begin{verbatim}
## [1] 8
\end{verbatim}

Note that the \texttt{echo\ =\ FALSE} parameter was added to the code
chunk to prevent printing of the R code that generated the plot.

\end{document}
